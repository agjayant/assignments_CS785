\documentclass{article}
\usepackage{geometry}
\usepackage{graphicx}
\usepackage{amsmath}
\usepackage{algorithm}
\usepackage{algpseudocode}
\usepackage{dsfont}
\usepackage{amssymb}
\usepackage{multicol}

\geometry{
a4paper,
right=10mm,
left=10mm,
top=10mm,
bottom=10mm,	
}
\newcommand\tab[1][15pt]{\hspace*{#1}} 
\begin{document}

\pagenumbering{gobble}

\begin{center}
\textbf{\Large CS785A : Multi-Agent Systems} \\
\textit{\large Jayant Agrawal}         14282
\end{center}

\textbf{\large Assignment 3}
\\ \\
\textbf{Solution 1.}\\
\textbf{(a).} Number of actions in round i is $(n-i)$. The total number of rounds before every player gets a share is $n$, since exactly one player gets a share in each round. Thus, total number of actions for the protocol: 
$$\sum_{i=1}^n (n-i)$$
This is of the order of $n^2$. \\ \\
\textbf{(b).} \\ \\ 
\textbf{(c).} Number of actions in round i (when $p_{i+1}$ enters) is $i(i+1)$, because the shares of $i$ players have to be divided into $(i+1)$ divisions each. Total number of rounds is $n$, since the protocol ends when all players have entered. Thus, the total number of actions for the protocol:
$$\sum_{i=1}^n i(i+1)$$
This is of the order of $n^3$.  \\ \\
\textbf{Solution 2.} \\
\textbf{(a).} If it is shown that the protocol is fair for all the three possibilities in step 3, then the protocol is fair. \\
$\tab$ \emph{\underline{Possibility 1}- }There are more than one fair share for $p_2$.  Also, $p_3$ has atleast one share which is fair, since the resource is divided into 3. Now, irrespective of the choice of $p_3$, there is still atleast one share left which is fair for $p_2$. $p_1$ gets the remaining share, which is fair by Step 1. If more than one share is fair for $p_2$ and $p_3$ both, then also the same thing holds.\\
\tab \emph{\underline{Possibility 2}- } $p_2$ and $p_3$ get fair allocation because their declared shares are different and there is no conflict. $p_1$ is indifferent for all allocations. Hence, everyone gets fair share. \\
\tab \emph{\underline{Possibility 3}- } Since, both declare the same share as fair, the share picked up by $p_1$ has a value less than $ 1/3 $ for both $p_2$ and $p_3$. So, the remaining shares combined must have value greater than $ 2/3 $ for both $p_2$ and $p_3$. Now, say $p_2$ divides the remaining share into 2 and $p_3$ gets to choose. Here again, there exists atleast one share for $p_3$ which is fair and $p_2$ is indifferent between the two.
Thus, the protocol is fair under all the three possibilities. \\ \\
\textbf{(b).} 

\end{document}


